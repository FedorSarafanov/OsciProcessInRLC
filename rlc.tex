\input{text/diss}
\usepackage{setspace}

\begin{document}

\def\labauthors{Понур К.А., Сарафанов Ф.Г., Сидоров Д.А.}
\def\labgroup{420}
\def\labnumber{217}
\def\labtheme{Исследование колебательных процессов в электрическом контуре}
\renewcommand{\vec}{\mathbf}
\renewcommand{\Re}{\operatorname{Re}}
\renewcommand{\Im}{\operatorname{Im}}
\renewcommand{\phi}{\varphi}
\renewcommand{\kappa}{\varkappa}
\renewcommand{\hat}{\widehat}
%%%%%%%%%%%%%%%%%%%%%%%%%%%%%%%%%%%%%%%%%%%%%%%%%%%%%%%%%%%%%%%%%%%%%%%%%%%%%%%
\input{text/titlepage}
%%%%%%%%%%%%%%%%%%%%%%%%%%%%%%%%%%%%%%%%%%%%%%%%%%%%%%%%%%%%%%%%%%%%%%%%%%%%%%%
\begin{spacing}{1}
\tableofcontents
\end{spacing}
% \setstretch{1.2}
\newpage
%%%%%%%%%%%%%%%%%%%%%%%%%%%%%%%%%%%%%%%%%%%%%%%%%%%%%%%%%%%%%%%%%%%%%%%%%%%%%%%


 \section{}
\subsection{Введение}
Цель работы -- экспериментальное исследование колебатель¬
ных процессов в линейном осцилляторе с потерями. В качестве осциллятора используется электрический
контур, состоящий из последовательно соединенных
катушки индуктивности $L$,
конденсатора $C$, резистора $R$
и внешнего источника ЭДС $\varepsilon$.
Дифференциальное уравнение, описывающее процессы в исследуемом контуре, имеет следующий вид:
\begin{equation}
	q''+2\delta q'+\omega_0 q=f(t),
\end{equation}
где q--заряд на конденсаторе, $\delta=\frac{R}{2L}$--коффициент затухания, $\omega_0=\frac{1}{\sqrt{LC}}$--собственная частота контура, $f(t)=\frac{\epsilon(t)}{L}$-- "вынужда.щая сила", $\epsilon(t)$-- внешняя ЭДС.
 

 С математической точки зрения уравнение (1) является неоднородным линейным дифференциальным уравнением 2-го порядка с постоянными коэффициентами. Решение такого уравнения, как известно, можно представить в виде суммы общего решения соответствующего однородного уравнения
\begin{equation}
	q''+2\delta q'+\omega_0 q=0
\end{equation}
и частного решения неоднородного. Уравнение (2) описывает поведение осциллятора в отсутствие внешней ЭДС, т.е. так назы¬ваемые собственные (свободные) колебания, а частное решение неоднородного уравнения (1) (в случае периодического внешнего воздействия) - вынужденные колебания. Исследованию этих двух режимов и уделяется основное внимание в работе. Кроме того, лабораторная установка позволяет наблюдать некоторые переходные процессы, в частности процессы установления колеба¬ний.
\begin{figure}[h!]
	\centering
    \includegraphics[]{chem/rcle}
    \caption{}
    \label{chem:rcle}
\end{figure}

%%%%%%%%%%%%%%%%%%%%%%%%%%%%%%%%%%%%%%%%%%%%%%%%%%%%%%%%%%%%%%%%%%%%%%%%%%%%%%%
\subsection{Собственные колебания в электрическром контуре}
При анализе решения уравнения (2) удобно выделить три случая: $\delta > \omega_0$,$\delta < \omega_0$ ,$\delta = \omega_0$.
В случае достаточно слабого затухания, когда $\delta < \omega_0$ общее решение уравнения (2) можно представить в виде
\begin{equation}
	q=A_0 e^{-\delta t}\cos(\omega_s t+\phi),
\end{equation}
где $\omega_s=\sqrt{\omega_0^2-\delta^2}$, а $A_0$ и $\phi$ -- произвольные постоянные, определяемые из начальных условий.
Процесс вида (3) называют затухающими квазигармоническими колебаниями. Если $\delta<<\omega_0$, то выличину $A(t)=A_0e^{-\delta t}$ можно считать медленно меняющейся амплитудой, а 
$T=\frac{2\pi}{\omega_s}$-- "периодом" этих колебаний.
В отсутстввие затухания $(\delta=0)$ решение уравнения (2) можно представить в виде
\begin{equation}
	q=A_1e^{\alpha_1 t}+A_2e^{\alpha_2 t},
\end{equation}
где $A_1$ и $A_2$ зависят от начальных условий, а 
\begin{equation}
	\alpha_{1,2}=-\delta+-\sqrt{\delta^2-\omega_0^2}
\end{equation}

Процесс, описываемый формулой (4), называется апериодическим.


Условие $\delta=\omega_0$ определяет критический режим колебаний, а соответствующее этому условию сопротивление называется кри 
тическим сопротивлением контура: $R_{\text{кр}}=2\sqrt{\frac{L}{C}}$. На практике используется также характеристическое (волновое) сопротивление 
контура: $\rho=\sqrt{\frac{L}{C}}$.
\subsection{Декремент затуъания. Добротность}
Логарифмическим декрементом затухания с! называется лога¬рифм отношения значений заряда с\ на пластинах конденсатора в двух последовательных (п-ом и (п+1)-ом) максимумах:
\newpage
\section{Заключение}

\end{document} 