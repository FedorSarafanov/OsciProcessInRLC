\input{text/diss}
\usepackage{setspace}

\begin{document}

\def\labauthors{Понур К.А., Сарафанов Ф.Г., Сидоров Д.А.}
\def\labgroup{420}
\def\labnumber{217}
\def\labtheme{Исследование колебательных процессов в электрическом контуре}
\renewcommand{\vec}{\mathbf}
\renewcommand{\Re}{\operatorname{Re}}
\renewcommand{\Im}{\operatorname{Im}}
\renewcommand{\phi}{\varphi}
\renewcommand{\kappa}{\varkappa}
\renewcommand{\hat}{\widehat}
%%%%%%%%%%%%%%%%%%%%%%%%%%%%%%%%%%%%%%%%%%%%%%%%%%%%%%%%%%%%%%%%%%%%%%%%%%%%%%%
\input{text/titlepage}
%%%%%%%%%%%%%%%%%%%%%%%%%%%%%%%%%%%%%%%%%%%%%%%%%%%%%%%%%%%%%%%%%%%%%%%%%%%%%%%
\begin{spacing}{1}
\tableofcontents
\end{spacing}
% \setstretch{1.2}
\newpage
%%%%%%%%%%%%%%%%%%%%%%%%%%%%%%%%%%%%%%%%%%%%%%%%%%%%%%%%%%%%%%%%%%%%%%%%%%%%%%%


 \section{}
\subsection{Введение}
Цель работы -- экспериментальное исследование колебатель¬
ных процессов в линейном осцилляторе с потерями. В качестве осциллятора используется электрический
контур, состоящий из последовательно соединенных
катушки индуктивности $L$,
конденсатора $C$, резистора $R$
и внешнего источника ЭДС $\varepsilon$.
Дифференциальное уравнение, описывающее процессы в исследуемом контуре, имеет следующий вид:
\begin{equation}
	q''+2\delta q'+\omega_0 q=f(t),
\end{equation}
где q--заряд на конденсаторе, $\delta=\frac{R}{2L}$--коффициент затухания, $\omega_0=\frac{1}{\sqrt{LC}}$--собственная частота контура, $f(t)=\frac{\epsilon(t)}{L}$-- "вынужда.щая сила", $\epsilon(t)$-- внешняя ЭДС.
 

 С математической точки зрения уравнение (1) является неоднородным линейным дифференциальным уравнением 2-го порядка с постоянными коэффициентами. Решение такого уравнения, как известно, можно представить в виде суммы общего решения соответствующего однородного уравнения
\begin{equation}
	q''+2\delta q'+\omega_0 q=0
\end{equation}
и частного решения неоднородного. Уравнение (2) описывает поведение осциллятора в отсутствие внешней ЭДС, т.е. так назы¬ваемые собственные (свободные) колебания, а частное решение неоднородного уравнения (1) (в случае периодического внешнего воздействия) - вынужденные колебания. Исследованию этих двух режимов и уделяется основное внимание в работе. Кроме того, лабораторная установка позволяет наблюдать некоторые переходные процессы, в частности процессы установления колеба¬ний.
\begin{figure}[h!]
	\centering
    \includegraphics[]{chem/rcle}
    \caption{}
    \label{chem:rcle}
\end{figure}

%%%%%%%%%%%%%%%%%%%%%%%%%%%%%%%%%%%%%%%%%%%%%%%%%%%%%%%%%%%%%%%%%%%%%%%%%%%%%%%
\subsection{Собственные колебания в электрическром контуре}
При анализе решения уравнения (2) удобно выделить три случая: $\delta > \omega_0$,$\delta < \omega_0$ ,$\delta = \omega_0$.
В случае достаточно слабого затухания, когда $\delta < \omega_0$ общее решение уравнения (2) можно представить в виде
\begin{equation}
	q=A_0 e^{-\delta t}\cos(\omega_s t+\phi),
\end{equation}
где $\omega_s=\sqrt{\omega_0^2-\delta^2}$, а $A_0$ и $\phi$ -- произвольные постоянные, определяемые из начальных условий.
Процесс вида (3) называют затухающими квазигармоническими колебаниями. Если $\delta<<\omega_0$, то выличину $A(t)=A_0e^{-\delta t}$ можно считать медленно меняющейся амплитудой, а 
$T=\frac{2\pi}{\omega_s}$-- "периодом" этих колебаний.
В отсутстввие затухания $(\delta=0)$ решение уравнения (2) можно представить в виде
\begin{equation}
	q=A_1e^{\alpha_1 t}+A_2e^{\alpha_2 t},
\end{equation}
где $A_1$ и $A_2$ зависят от начальных условий, а 
\begin{equation}
	\alpha_{1,2}=-\delta+-\sqrt{\delta^2-\omega_0^2}
\end{equation}

Процесс, описываемый формулой (4), называется апериодическим.


Условие $\delta=\omega_0$ определяет критический режим колебаний, а соответствующее этому условию сопротивление называется кри 
тическим сопротивлением контура: $R_{\text{кр}}=2\sqrt{\frac{L}{C}}$. На практике используется также характеристическое (волновое) сопротивление 
контура: $\rho=\sqrt{\frac{L}{C}}$.

%%%%%%%%%%%%%%%%%%%%%%%%%%%%%%%%%%%%%%%%%%%%%%%%%%%%%%%%%%%%%%%%%%%%%%%%%%%%%%%
\subsection{Декремент затухания. Добротность}
Логарифмическим декрементом затухания $d$ называется логарифм отношения значений заряда $q$ на пластинах конденсатора в двух последовательных ($n$-ом и ($n+1$)-ом) максимумах:
\begin{equation}
	d=\ln{\frac{q_n}{q_{n+1}}}=\delta T
\end{equation}
Поскольку, как видно из выражения (3), $\delta$ есть величина, обратная промежутку времени $\tau$, за которое амплитуда колебаний спадает в $e$ раз, то можно определить число колебаний $N$. совершившихся за это время:
\begin{equation}
	N=\frac{\tau}{T}=
		\frac{1}{\delta T}=
			\frac{1}{d}
\end{equation}

Таким образом, логарифмический декремент затухания $d$ есть величина, обратная числу колебаний N. в течение которых амплитуда колебаний уменьшается в $e$ раз.
Часто для характеристики затухания удобнее использовать не N. а величину в $\pi$ раз большую - добротность контура $Q=\pi N$.
 
При малом затухании ($\delta \ll \omega_0$) частота собственных колебаний 
$\omega_s=\sqrt{\omega_0^2 - \delta^2}=\omega_0$. 
При этом добротность контура и его
логарифмический декремент затухания можно выразить через параметры контура L, С, R следующим образом:
\begin{equation}
	d=\frac{\pi R}{\omega_0 L}=\pi RC\omega_0=\pi R\sqrt{\frac{C}{L}}
\end{equation}
\begin{equation}
	Q=\frac{\omega_0 L}{R}=\frac{1}{RC\omega_0}=\frac{1}{R}\sqrt{\frac{L}{C}}
\end{equation}

Необходимо иметь в виду, что во всех этих выражениях под следует понимать сопротивление, эквивалентное всем потерям в контуре. Дополнительные потери при прохождении переменного тока могут быть вызваны гистерезисом и токами Фуко в сердеч¬нике катушки индуктивности, токами утечки и процессами поля¬ризации в диэлектрике конденсатора.
%%%%%%%%%%%%%%%%%%%%%%%%%%%%%%%%%%%%%%%%%%%%%%%%%%%%%%%%%%%%%%%%%%%%%%%%%%%%%%%
\subsection{Фазовая плоскость} % (fold)

Процессы в колебательном контуре удобно изображать на так называемой фазовой плоскости, где по оси абсцисс откладывают заряд $q$, а по оси ординат величину, пропорциональную току.
например $\frac{\dot{q}}{\omega_0}$ (это величины одной размерности). Каждому
состоянию колебательного контура, характеризуемому мгновенными значениями $q$ и $\dot{q}$, соответствует точка на фазовой плоско¬сти (изображающая точка). Изменение состояния вызывает перемещение изображающей точки по фазовой плоскости. Линия, описываемая изображающей точкой, называется фазовой траекторией. Совокупности движений с разными начальными условиями соответствует семейство фазовых траекторий. Например, гармоническим колебаниям в контуре без затухания  на фазовой плоскости соответствует семейство окружностей с общим центром в начале координат. Свободные затухающие колебания в контуре изображаются фазовыми траекториями в виде скручивающихся к началу координат спиралей.
%%%%%%%%%%%%%%%%%%%%%%%%%%%%%%%%%%%%%%%%%%%%%%%%%%%%%%%%%%%%%%%%%%%%%%%%%%%%%%%
\subsection{Вынужденные колебания в электрическом контуре} % (fold)
Колебания в контуре под действием внещней гармонической силы описывается уравнением
\begin{equation}
	\ddot{q}+2\delta\dot{q}+\omega_0^2 q= F_0\cos{\omega t}
\end{equation}
Решение этого уравнения, соответствующее установившемуся режиму, имеет вид
\begin{equation}
	q(t)=B(\omega)\cos{\omega t+\psi},
\end{equation}
где амплитуда $B(\omega)$ и фза $\psi$ определяются следующим образом:
\begin{equation}
	B(\omega)=\frac{F_0}{\sqrt{(\omega^2-\omega_0^2)^2+4\delta^2\omega^2}}
\end{equation}
\begin{equation}
	\tan{\psi}=\frac{2\delta\omega}{\omega^2-\omega_0^2}
\end{equation}
	Используя формулы (11), (12), (13) и учитывая, что $F-0=\frac{\epsilon_0}{L}$
Нетрудно получить выражения для амплитуды тока в контуре
\begin{equation}
	I_0=\omega B(\omega)=\frac{\epsilon_0}{\sqrt{R^2+(\omega L-\frac{1}{\omega C})^2}}
\end{equation}
и для амплитуд напряжений на отдельных элементах контура
\begin{equation}
	U_L=I_0\omega L=\frac{\omega L\epsilon_0}{\sqrt{R^2+(\omega L-\frac{1}{\omega C})^2}}
\end{equation}
\begin{equation}
	U_C=\frac{I_0}{\omega C}=\frac{\epsilon_0}{\sqrt{R^2+(\omega L-\frac{1}{\omega C})^2}}
\end{equation}
\begin{equation}
	U_R=I_0R=\frac{\epsilon_0R}{\sqrt{R^2+(\omega L-\frac{1}{\omega C})^2}}
\end{equation}

\newpage
\section{Экспериментальная часть}
\subsection{Исследование свободных колебаний}
На установке получены осциллограммы $U_c(t)$ и $I(t)$ при нескольких сопротивлениях контура $R$. 

Нашли период собственных колебаний контура и декремент затухания d для различных R

Построить графики зависимости $Т$ и $d$ от $R$.

 По выполненным измерениям рассчитать добротность контура О и коэффициент затухания 5 для одного из значений R. Вычислить индуктивность и критическое сопротивление контура.
с) Получить на экране осциллографа и зарисовать несколько фазовых траекторий при различных R. Напряжение, пропорциональое току в контуре, снимается с сопротивления К = 250 Ом и подается на вход Y осциллографа. Напряжение Uc подается на вход X. Развертка осциллографа должна быть выключена.
\subsection{Исследование вынужденных колебаний}
\subsection{Исследование процессов установления вынужденных колебаний}
\section{Заключение}

\end{document} 